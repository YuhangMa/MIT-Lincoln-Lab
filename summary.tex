In this paper we examined time series data  regarding the number of Ebola cases and deaths in Liberia during 2014-2015 outbreak. Through system dynamics approach we showed that intervention can have a significant impact on the spread of the disease. A change in model parameters, caused by international intervention, decreased Ebola's basic reproductive number from 1.99 to 0.787. In the long run, a disease would spread if it was not for the intervention. Because system dynamics is a deterministic approach, we also looked at an agent-based variation of the model. Through different uncertainty realizations, we saw that  70\% of the time Ebola would spread to more than 2\% of the population and when it does, the consequence are rather severe: half of the population dies because of the disease. Through the probabilistic model we also saw the effect of intervention. If the intervention is started early on, when only 1\% of the population is exposed to the disease, only 2\% of the population will die because of it.  \textbf{EMILY AND YUHANG RESULTS
NEED TO ADD SUMMARY!!}

There are several factors one may wish to take into account when analyzing how a virus spreads in a community. While we took into account characteristics of the virus itself: incubation period, the rate of recovery, mortality rate and society's cultural perspective on a proper burial ceremony, other customs and beliefs play an important role. These include frequency and nature of individuals' interactions, distance and frequency of travel to other villages, cities or countries and transportation utilized to name a few. One possible extension of this work is to add a network structure and examine how the disease spreads based on the connectivity of the network.

Having information about the local government and the wealth of the region, would help to determine the capacity of response when facing an epidemic; including the quantity and quality of hospitals, their capacity and the number of healthcare professionals as well as 
their level of  expertise. While some data regarding these aspects is available, it is rather scattered and we did not spend much time pursuing this objective. 

In the moment of a virus outbreak, governments from other countries may intervene to help to control the disease, possibly decreasing the number of infected people and increasing the number of recovered patients, by educating individuals about the virus and its mode of transmission as well as properly handling the deceased family. One realization of the result of such action we saw during 2014-2015 Ebola outbreak in Liberia. Future work will concentrate on extending the social parameters. 

