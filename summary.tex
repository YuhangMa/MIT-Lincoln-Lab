In this paper we examined time series data of the number of Ebola cases and deaths in Liberia during the 2014-2015 outbreak. Through a system dynamics approach, we showed that intervention can have a significant impact on the spread of the disease. A change in model parameters, caused by international intervention, decreased Ebola's reproductive number from 1.99 to 0.787. Because system dynamics is a deterministic approach, we also looked at an agent-based variation of the model. Through different uncertainty realizations, we saw that in 70\% of instances Ebola would spread to more than 2\% of the population with severe consequences: half of the population dies. Through the probabilistic model we also studied the effect of intervention. If the intervention is started early, when only 1\% of the population is exposed to the disease, only 2\% of the population dies. We then incorporated spatial movement into an agent-based model and allowed for different types of individuals (family vs. community) and local transmission of the infection. In this setting, the outbreaks can be less severe due to the environment not being well-mixed. When tracking how many infections result from different types of contact, the spread between community members accounts for the highest proportion of transmission. The transmission during funerals is still significant, however, and reducing the common practices of touching the deceased at a funeral would lower the number of infected individuals.

There are several factors one may wish to take into account when analyzing how a virus spreads in a community. We took into account characteristics of the virus itself: incubation period, the rate of recovery, mortality rate and society's cultural perspective on a proper burial ceremony. In the spatial model we also attempted to incorporate frequency and nature of individuals' interactions, and distance and frequency of travel to other villages. Other factors also play a role in the spread of the disease. Having information about the local government and the wealth of the region would help to determine the capacity of response when facing an epidemic, including the quantity and quality of hospitals, their capacity, and the number of health-care professionals as well as their level of expertise. In the moment of a virus outbreak, governments from other countries may intervene to help to control the disease, possibly decreasing the number of infected people and increasing the number of recovered patients, by educating individuals about the virus and its mode of transmission as well as properly handling the deceased family. One realization of the result of such action we saw during 2014-2015 Ebola outbreak in Liberia. Such social parameters could be incorporated into our models.