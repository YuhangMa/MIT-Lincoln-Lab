The massive outbreak of Ebola in Africa in the past two years has spiked interest in modeling the spread of disease to the purpose of predicting such outbreaks and how to prevent them in the future. In this paper, we focus on the spread of Ebola through Liberia in 2014-2015, due to the availability of relevant data for Liberia in the literature. Using properly calibrated parameters, our models can be used to consider other countries or diseases to predict and prevent outbreaks.

Ebola was first discovered in 1976. Between 1976 and 2014, there were 34 recorded cases of Ebola reported by the Center for Disease Control and Prevention. Until 2014, only small outbreaks occurred, with less than 500 reported deaths per outbreak -- 11 such outbreaks had less than 10 recorded deaths, and 15 outbreaks had less than 65 deaths \cite{CDCOutbreaks}. In contrast, the 2014 outbreak has taken thousands of lives over the past two years and has received extensive media coverage.

Earlier outbreaks of Ebola were contracted in Central Africa, where it was first discovered, but the current outbreak took place in West Africa. The early symptoms are flu-like -- fever, headache, fatigue, and joint pain -- which may have initially masked the identity of the virus, because other common diseases to the area, such as HIV and Malaria, share these symptoms. As the Ebola virus progresses through an individual, they experience abdominal pain, diarrhea, vomiting, and rashes. The virus is contracted through direct contact with bodily fluids (e.g. blood, saliva, urine, fecal matter) and physical objects contaminated with such fluids, and once the virus is contracted, the incubation period may last up to two weeks making prevention measures like tracking difficult \cite{CDCSympt}. A variety of cultural and economic factors have also contributed to the spread of the disease: lack of medical centers and poor sanitation practices in such centers, distrust in western medicine, poverty of local governments and individuals, and physical contact with the deceased in traditional burial ceremonie s\cite{WHOReasons}.

We model the spread of Ebola with both system-based and agent-based models. In the system-based model, the population is divided into compartments. As time progresses, people flow from one compartment into another. These dynamics are described by difference or differential equations. One of the most well-known system-based models is the SIR model which involves three states: susceptible, infected, recovered. From this, other models are generated with more compartments. In the case of Ebola, Lekone and Finkenstäd \cite{Lekone2006} consider a four-compartment model, inserting an ``exposed" state between the susceptible and infected states. A six-compartment model, presented by Legrand \cite{Legrand2007} and Rivers \cite{Rivers2014}, differentiates between the disease spread rates in the community, at hospitals and medical centers, and at funerals.

In contrast to system-based models, agent-based models are concerned with the behavior of a typical individual and probabilistic events rather than the deterministic behavior of the overall system. In such models, each individual has a state and they transition to other states probabilistically at each time step. Agent-based models for Ebola in the previous literature include Siettos et al. \cite{Siettos2015} and Merler et al.\cite{Merler2015}. A variety of models examine the effectiveness of intervention measures, including contact tracing by Webb et al. \cite{Webb2015} and incorporating travel restrictions by Poletto et al. \cite{Poletto2014}.

This paper presents a seven-compartment model, used in both the system and agent-based dynamics. We show that an intervention can cause the trajectory of the spread of the disease to change significantly. The deterministic model predicts that over 50\% of the population who would have died from the disease never contract it due to intervention. The probabilistic model also reviles similar results. 

In Section 3 we present the seven-compartment model and discuss assumptions and parameters used throughout the paper. Section 4 presents the system-based model and calibrates parameters used for the rest of the paper. Sections 5 and 6 present agent-based model, the former based on the flows of the system-based model and the latter incorporating spatial movement of individuals. In Section 7  we summarize the results and discuss future extensions of the models and approaches.  