\begin{itemize}
\item We consider a model of Ebola Outbreak with parameters calibrated for Liberia.
\item We consider two time periods. The first one starts with the announcement of Ebola outbreak in March 2014 and ends the day of the International Intervention in September 2014. The second period covers the time from the International Intervention to present.
\item We ignore all the possible births and deaths occurred due to reasons other that Ebola during the chosen time. 
\item Each individual who dies because of Ebola has a funeral.
\end{itemize}


\subsection{Factors Considered}
There are several factors to take into account when analyzing how a virus may spread in a community. First of all the customs and beliefs of a society play an important role, it implies how the individuals interact among themselves, how often they visit their relatives or friends, the amount of travel to other villages, cities or countries to work or buy supplies, as well as the mode of transport.\\
Having information about the local government and the wealth of the region, would help to determine the capacity of response when facing an epidemic; including the quantity and quality of hospitals and their capacity, as well as the amount of healthcare workers and their expertise.\\
In the moment of a virus outbreak, governments from other countries may intervene to help to control the disease, possibly decreasing the number of infected people and increased the number of recovered patients, by educating individuals about the virus, the ways of transmission and handling of the deceased family. \\
Characteristics of the virus itself are also important, having an estimate of the incubation period, the rate of recovery, the time it remains in the deceased would help to predict the behavior of the virus.\\


\subsection{Compartment-State Definition}
\begin{description}
\item[S]- Susceptible. Individuals who have not contracted the disease and have no immunity to it. 
\item [E] - Exposed. Individuals who have come in contact with the Ebola patient and have contracted the disease but do not yet exhibit sever symptoms and thus, are considered not infectious.
\item [I] - Infected. Individuals who experience severe symptoms of Ebola and are contagious.
\item [H] - Hospitalized. Individuals who are infectious and are in the hospital because they are experiencing severe symptoms of Ebola.
\item[F] - Funeral. Diseased but still contagious victims of Ebola. 
\item[D] - Dead. Individuals who died because of Ebola, were buried and are currently under ground. They are considered not to be contagious.
\item [R] - Recovered. Individuals who had Ebola, survived and now are immune to the disease.
\end{description}

